
\section{Fonctions à valeurs scalaires}

\vspace{0.5cm}

\begin{center}
    \( \KK\) désigne indifféremment \(\RR\) ou \(\CC\).
\end{center}


\subsection{Généralités}

\vspace{0.2cm}

\begin{center}
Soit E un ensemble. 
\end{center}

\vspace{0.6cm}

\noindent Soit \(f : E\to \KK\ \) une application.
\vspace{-0.2cm}
\begin{itemize}
    \item[•] \(f\) est dite \textbf{bornée}\index{fonctions à valeurs scalaires!bornée} sur E \ssi : \(\ \exists M\in \RR ^+, \ \forall x\in E, \ |f(x)|\leq M\). \\
    \begin{small}L'ensemble des applications bornées de E dans \(\KK\) est noté B(E,\(\ \KK\)).\end{small}\vspace{0.3cm}
    
    \item[•] Lorsque \(f\) est à valeurs \underline{réelles} (i.e. \(\KK=\RR\)), \\
    \(f\) est dite \textbf{majorée}\index{fonctions à valeurs scalaires!majorée} sur E \ssi : \(\ \exists M\in \RR \ | \ \forall x \in E,\ f(x)\leq M.\) \\
    \(f\) est dite \textbf{minorée}\index{fonctions à valeurs scalaires!minorée} sur E \ssi : \(\ \exists m\in \RR \ | \ \forall x \in E,\ f(x)\geq m.\)\vspace{0.3cm}
     
    \item[•] Lorsque \(f\) est une application d'une variable \underline{réelle} et à valeurs \underline{réelles} \\
    (i.e. \(\KK=\RR\) et E=\(\RR\)), \\
    \(f\) est dite \textbf{croissante}\index{fonctions à valeurs scalaires!croissante} sur E \ssi : \(\ \forall (x,y)\in E^2,\ \ x\leq y\ \Rightarrow \ f(x)\leq f(y).\) \\
    \(f\) est dite \textbf{décroissante}\index{fonctions à valeurs scalaires!décroissante} sur E \ssi : \(\ \forall (x,y)\in E^2,\ \ x\leq y\ \Rightarrow \ f(x)\geq f(y).\) \\
    \(f\) est dite \textbf{monotone}\index{fonctions à valeurs scalaires!monotone} sur E \ssi \(f\) est décroissante ou croissante sur E.\vspace{0.3cm}
    
    \item[•] Lorsque E\(\ \subset \KK\), on appelle \textbf{période}\index{période} de \(f\) tout élément T de E vérifiant :\vspace{-0.3cm}
    \begin{center}\(\ \forall x\in E,\ f(x+T)=f(x).\)\end{center}\vspace{-0.3cm}
    \begin{small}L'ensemble des périodes de l'application \(f\) est noté P(\(f\)).\end{small} \\
    On dit que \(f\) est \textbf{périodique}\index{fonction périodique} \ssi \(f\) admet au moins une période non nulle.\vspace{0.3cm}

    
    \item[•] Lorsque E est un \underline{intervalle} de $\RR$, et \(f\) est à valeurs \underline{réelles} (i.e. \(f : E\to \RR\ \)),\vspace{0.1cm} \\
    \(f\) est dite \textbf{convexe}\index{fonction convexe} sur E \ssi : \( \ \forall (x,y) \in E^2,\ \forall \lambda \in [0,1],\) on a : \vspace{-0.3cm} 
    \begin{center}\(f\bigl((1-\lambda)x+\lambda y\bigr) \leq (1-\lambda)f(x) + \lambda f(y).\) \end{center} 
    \(f\) est dite \textbf{concave}\index{fonction concave} sur E \ssi :\( \ \forall (x,y) \in E^2,\ \forall \lambda \in [0,1],\) on a : \vspace{-0.3cm} 
    \begin{center}\((1-\lambda)f(x) + \lambda f(y) \leq f\bigl((1-\lambda)x+\lambda y\bigr).\) \end{center}
\end{itemize}

\vspace{1cm}

Soient \(\,f:I\to\KK\,\) une application et \(a\in I\). On dit que $f$ est \textbf{continue au point a}\index{fonctions à valeurs scalaires!continue en un point}\footnote{Sinon on dit que $a$ est un point de discontinuité de $f$} \ssi \(\,\displaystyle \adjustlimits\lim_{t\,\to\,a} f(t)=f(a)\).\vspace{0.2cm}\\
On dit que $f$ est \textbf{continue sur I}\index{fonctions à valeurs scalaires!continue sur un intervalle} \ssi $f$ est continue en tout point de I.

\vspace{1cm}

\noindent Soient \(\,f:I\to \RR\,\) une application et \(a\in I\).
\begin{itemize}[leftmargin=1cm, label=•]
	\item On dit que $f$ est \textbf{continue à gauche}\index{fonctions à valeurs scalaires!continue à gauche} en $a$ lorsque \(\displaystyle \,I\,\cap\;]-\infty,a[\;\neq \varnothing \ \text{ et } \lim_{\ \,t\,\to\, a^-}f(t)=f(a)\).
	
	\item On dit que $f$ est \textbf{continue à droite}\index{fonctions à valeurs scalaires!continue à droite} en $a$ lorsque \(\displaystyle \,I\,\cap\;]a,+\infty[\;\neq \varnothing \ \text{ et } \lim_{\ \,t\,\to\, a^+}f(t)=f(a)\). 

\end{itemize}

\vspace{1cm}

On dit que \(\,f:[a,b]\to \KK\,\) est \textbf{en escalier}\index{application en escalier} sur $[a,b]\,$ \ssi il existe une subdivision\\
\(\sigma=(a_i)_{i\in \llbracket 0,n \rrbracket}\,\) de $[a,b]\,$ telle que pour tout \(i\in \llbracket 1,n \rrbracket\),\, $f\,$ soit constante sur l'intervalle ouvert \(]a_{i-1},\, a_i[\,\).\vspace{0.1cm}\\
Une telle subdivision $\sigma$ est dite \textbf{adaptée}\index{fonctions à valeurs scalaires!subdivision adaptée} à $f$.\vspace{0.1cm}\\
\begin{small}
    L'ensemble des applications de $[a,b]\,$ dans $\,\KK\,$ qui sont en escalier sur $[a,b]\,$ est noté \(\,\mathcal{E}\bigl([a,b]\,,\KK\bigr).\)
\end{small}

\vspace{1cm}

On dit que \(\,f:[a,b]\to \KK\,\) est \textbf{continue par morceaux}\index{fonctions à valeurs scalaires!continue par morceaux sur un segment} sur \([a,b]\,\) \ssi il existe une subdivision \(\sigma=(a_i)_{i\in \llbracket 0,n \rrbracket}\,\) de $[a,b]\,$ telle que pour tout \(i\in \llbracket 1,n \rrbracket\),\, la restriction de $f$ à l'intervalle ouvert \(]a_{i-1},\,a_i[\,\) se prolonge en une application continue sur le segment \([a_{i-1},\,a_i].\)\vspace{0.1cm}\\
Une telle subdivision $\sigma$ est dite \textbf{adaptée}\index{fonctions à valeurs scalaires!subdivision adaptée} à $f$.\vspace{0.1cm}\\
\begin{small}
    L'ensemble des applications de \([a,b]\,\) dans \(\,\KK\,\) qui sont continues par morceaux sur \([a,b]\,\) est noté\, M\(^0\bigl([a,b]\,\KK\bigr).\)
\end{small}

\vspace{1.3cm}

Soient I un intervalle d'intérieur non vide de $\,\RR\,$, \(\,f:I\to \KK,\ a\in \KK\,\) et \(\,T_a:I\setminus \{a\}\to \KK\,\) définie par : \(\,\displaystyle T_a(t)=\frac{f(t)-f(a)}{t-a}.\)\vspace{0.1cm}
\begin{itemize}[leftmargin=1cm]
    \item[•] On dit que $f$ est \textbf{dérivable en un point}\index{fonctions à valeurs scalaires!dérivable au point} $a$ \ssi \(\;\displaystyle \lim_{a,\,\neq}\,T_a\:\) existe dans $\,\KK\,$.
    
    \item[•] Si $a$ n'est pas l'extrémité droite de I, on dit que $f$ est \textbf{dérivable à droite}\index{fonctions à valeurs scalaires!dérivable à droite} en $a$ \ssi \(\;\displaystyle \lim_{a^+}\,T_a\;\) existe dans $\,\KK\,$. En cas d'existence, cette limite est noté \(f_d'(a)\) et est appelée dérivé à droite de $f$ en $a$.
    
    \item[•] Si $a$ n'est pas l'extrémité gauche de I, on dit que $f$ est \textbf{dérivable à gauche}\index{fonctions à valeurs scalaires!dérivable à gauche} en $a$ \ssi \(\;\displaystyle \lim_{a^-}\,T_a\;\) existe dans $\,\KK\,$. En cas d'existence, cette limite est noté \(f_g'(a)\) et est appelée dérivé à gauche de $f$ en $a$.
\end{itemize}

\vspace{1cm}

Une application \(\,f:I\to \KK\,\) est dite \textbf{dérivable sur l'intervalle}\index{fonctions à valeurs scalaires!dérivable sur un intervalle} I \ssi $f$ est dérivable en tout point de I. On note \(\,\mathcal{D}(I,\KK)\,\) l'ensemble des applications de I dans $\,\KK\,$ dérivables sur I.\\
Lorsque $f$ est dérivable sur I on dispose de l'application \(\, f':I\to \KK\,\) qui à tout réel \(t\in I\) associe le scalaire \(f'(t)\) de \(\,\KK\). L'application $f'$ est appelée \textbf{application dérivée}\index{fonctions à valeurs scalaires!application dérivée} de $f$

\vspace{1.3cm}

Soit \(\, f:I\to \KK.\,\) Par convention $f$ est dite zéro fois dérivable sur I et on pose : $f^{(0)}=f$.\vspace{0.1cm}\\
L'application $f$ est dite une fois dérivable sur I \ssi $f$ est dérivable sur I et\\
on pose alors : $f^{(1)}=f'$.\vspace{0.1cm}\\
Pour \(n\in \NN\)\expo{*}, l'application $f$ est dite \textbf{n-fois dérivable}\index{fonctions à valeurs scalaires!n-fois dérivable} sur I \ssi $f$ est $n-1$ fois dérivable sur I et si $f^{(n-1)}$ est dérivable sur I. Dans ces conditions on pose : \(f^{(n)}=\left(f^{(n-1)}\right)'.\)\vspace{0.1cm}\\
En cas d'existence, l'application $f^{(n)}$ est appelée \textbf{dérivée n}\expo{ième}\index{fonctions à valeurs scalaires!dérivée n\expo{ième}} de $f$. L'application $f$ est dite \textbf{indéfiniment déviable}\index{fonctions à valeurs scalaires!indéfiniment déviable} sur I \ssi pour tout \(n\in \NN\), $f$ est $n$-fois dérivable sur I. 

\vspace{1.3cm}

Soient \(\,f:I\to \KK\,\) une application et \(\,a\in I\). Si $f$ est $n$-fois dérivable en $a$ alors on appelle \textbf{Polynôme de Taylor}\index{Polynôme de Taylor} de $f$ en $a$ à l'ordre $n$ la fonction polynomiale notée $\,T_{n,f,a}\,$ définie par : \vspace{-0.1cm}
\[T_{n,f,a}(x)=\sum_{k=0}^n\frac{f^{(k)}(a)}{k!}(x-a)^k\]
La différence \(\,f-T_{n,f,a}\,\) est notée \(\,R_{n,f,a}\,\) et appelée reste de $f$ en $a$ à l'ordre $n$.

\vspace{2cm}

\subsection{Intégration}

\vspace{1cm}

\subsubsection{Intégration sur un segment}

\vspace{1cm}

On considère deux réels $a$ et $b$ tels que $a\leq b$. Si \(\,f\in \text{M}^0\bigl([a,b]\,\KK\bigr)\,\) alors l'intégrale de $f$ sur le segment \([a,b]\,\) est noté \(\,\displaystyle\int_{[a,b]}\!f\).

\vspace{1cm}

\underline{\emph{Théorème - définition}} : \textbf{Sommes de Riemann.}\index{somme de Riemann}\vspace{0.1cm}\\
Si \(\,f\in \text{M}^0\bigl([0,1],\KK\bigr)\,\) alors \(\,\displaystyle \lim_{n\to +\infty}\frac{1}{n}\sum_{k=1}^{n}f\!\left(\frac{k}{n}\right)=\lim_{n\to +\infty}\frac{1}{n}\sum_{k=0}^{n-1}f\!\left(\frac{k}{n}\right)=\int_{[0,1]}f.\)\vspace{0.3cm}\\
Si \(\,f\in \text{M}^0\bigl([a,b],\KK\bigr)\,\) alors \(\displaystyle \lim_{n\to +\infty}\frac{b-a}{n}\sum_{k=1}^{n}f\!\left(a+k\,\frac{b-a}{n}\right)=\lim_{n\to +\infty}\frac{b-a}{n}\sum_{k=0}^{n-1}f\!\left(a+k\,\frac{b-a}{n}\right)=\int_{[a,b]}f.\)

\vspace{1.5cm}

Soient $I$ un intervalle de \(\,\RR\,\) et \(\,f:I\to \KK\,\) une application. On dit que $f$ admet une \textbf{primitive}\index{fonctions à valeurs scalaires!primitive} sur $I$ \ssi il existe une application \(\,F:I\to \KK\,\) dérivable sur $I$ et telle que : \vspace{0.1cm}\\
\(\forall t\in I,\ F\,'(t)=f(t).\quad \) On dit alors que $F$ est \underline{une} primitive de $f$ sur $I$.

\newpage

\subsubsection{Intégrale généralisée}

\vspace{0.8cm}

\begin{center}
    Soient I un intervalle non vide de \(\,\RR\,\) et \((a,b)\in \overline{\RR}^{\,2}\,\) tel que $a<b$.
\end{center}

\vspace{0.7cm}

On suppose \(a\in\RR\ \) (mais $b$ peut valoir \(+\infty\)).\vspace{0.2cm}\\
On considère \(\,f:[a,\,b[\;\to\KK\,\) continue par morceaux sur l'intervalle \([a,\,b[\,\) et on lui associe l'application\vspace{0.1cm}\\
\(\,F:[a,\,b[\;\to\KK\;\) définie par : \(\,\displaystyle F(x)=\int_{a}^{x}f\).
\begin{itemize}[leftmargin=0.5cm]
    \item[•] On dit que l'intégrale de $f$ sur \([a,\,b[\,\) est \textbf{convergente}\index{intégrale convergente} \ssi \(\,\underset{b}{\lim}\,F\,\) existe dans \(\,\KK.\)\\
    Lorsque cette condition est remplie, on pose : \(\,\displaystyle \int_{a}^{b}\!f=\lim_{x\to b}\,\int_{a}^{x}\!f\).\\
    Le scalaire \(\,\displaystyle \int_{a}^{b}\!f\,\) est alors appelé \textbf{intégrale généralisée de $f$ sur}\index{intégrale généralisée}\footnote{Une intégrale généralisée est aussi appelée \textbf{intégrale impropre}\index{intégrale impropre}.} \(\,[a,\,b[\,.\)\vspace{0.2cm}

    \item[•] Pour exprimer que l'intégrale de $f$ sur \([a,\,b[\,\) ne converge pas on dit qu'elle \textbf{diverge}\index{intégrale divergente}.\vspace{0.4cm}
    
    \item[•] On suppose que l'intégrale de $f$ sur \([a,\,b[\,\) est convergente.\vspace{-0.2cm}
    
    \hspace{4cm}Soit \(R:[a,\,b[\;\to\KK\,\) définie par : \(\,\displaystyle R(x)=\int_{x}^{b}\!f\).\vspace{0.1cm}\\
    La quantité \(R(x)\) est appelée \textbf{reste d'ordre}\index{reste d'ordre $x$ d'une intégrale généralisée} $x$ de l'intégrale généralisée de $f$ sur \([a,\,b[\,\).
\end{itemize}

\vspace{2cm}

On suppose \(b\in\RR\ \) (mais $a$ peut valoir \(-\infty\)).\vspace{0.2cm}\\
On considère \(\,f:\;]a,\,b]\to\KK\,\) continue par morceaux sur l'intervalle \(\,]a,\,b]\,\) et on lui associe l'application\vspace{0.1cm}\\
\(\,F:\;]a,\,b]\to\KK\;\) définie par : \(\,\displaystyle F(x)=\int_{x}^{b}f\).
\begin{itemize}[leftmargin=0.5cm]
    \item[•] On dit que l'intégrale de $f$ sur \(\,]a,\,b]\,\) est \textbf{convergente}\index{intégrale convergente} \ssi \(\,\underset{a}{\lim}\,F\,\) existe dans \(\,\KK.\)\\
    Lorsque cette condition est remplie, on pose : \(\,\displaystyle \int_{a}^{b}\!f=\lim_{x\to a}\,\int_{x}^{b}\!f\).\\
    Le scalaire \(\,\displaystyle \int_{a}^{b}\!f\,\) est alors appelé \textbf{intégrale généralisée de $f$ sur}\index{intégrale généralisée} \(\,]a,\,b]\,.\)\vspace{0.2cm}

    \item[•] Pour exprimer que l'intégrale de $f$ sur \(\,]a,\,b]\,\) ne converge pas on dit qu'elle \textbf{diverge}\index{intégrale divergente}.\vspace{0.4cm}
    
    \item[•] On suppose que l'intégrale de $f$ sur \(\,]a,\,b]\,\) est convergente.\vspace{-0.2cm}
    
    \hspace{4cm}Soit \(R:\;]a,\,b]\to\KK\,\) définie par : \(\,\displaystyle R(x)=\int_{a}^{x}\!f\).\vspace{0.1cm}\\
    La quantité \(R(x)\) est appelée \textbf{reste d'ordre}\index{reste d'ordre $x$ d'une intégrale généralisée} $x$ de l'intégrale généralisée de $f$ sur \(\,]a,\,b]\,\).
\end{itemize}

\newpage

On ne suppose rien sur $a$ et $b$, on peut donc avoir \(\,a=-\infty\,\) et/ou \(\,b=+\infty\).\vspace{0.2cm}\\
Soit \(\,f:\;]a,\,b[\;\to\KK\,\) continue par morceaux sur l'intervalle \(\,]a,\,b[\,\).
\begin{itemize}[leftmargin=0.5cm]
    \item[•] On dit que l'intégrale de $f$ sur \(\,]a,\,b[\,\) est \textbf{convergente}\index{intégrale convergente} \ssi il existe \(\,c\in\,]a,\,b[\,\) tel que l'intégrale de $f$ sur \(\,]a,c]\,\) \underline{et} l'intégrale de $f$ sur \([c, b[\,\) sont convergentes.\vspace{0.1cm}\\
    Lorsque cette condition est remplie on pose : \(\,\displaystyle \int_{a}^{b}\!f\,=\int_{a}^{c}\!f+\int_{c}^{b}\!f.\)\vspace{0.1cm}\\
    Le scalaire \(\,\displaystyle\int_{a}^{b}\!f\,\) est appelé \textbf{intégrale généralisée de $f$ sur}\index{intégrale généralisée} \(\,]a,\,b[\,\) et il est indépendant du réel $c$ considéré dans \(\,]a,\,b[\,\).\vspace{0.4cm}

    \item[•] Pour exprimer que l'intégrale de $f$ sur \(\,]a,\,b[\,\) ne converge pas on dit qu'elle \textbf{diverge}\index{intégrale divergente}.
\end{itemize}

\vspace{1.5cm}

\noindent On suppose $I$ ouvert ou semi-ouvert. Soit \(\,f:I\to\KK\,\) continue par morceaux sur $I$.
\begin{itemize}
    \item[•] L'intégrale de $f$ sur $I$ est dite \textbf{absolument convergente}\index{intégrale absolument convergente} \ssi l'intégrale de $|f|$ sur $I$ est convergente.
    
    \item[•] L'intégrale de $f$ sur $I$ est dite \textbf{semi-convergente}\index{intégrale semi-convergente} \ssi l'intégrale de $f$ sur $I$ est convergente et non absolument convergente.
\end{itemize}

\vspace{1.5cm}

Une application \(\,f:I\to\KK\,\) est dite \textbf{intégrable}\index{application intégrable} sur $I$ \ssi elle est continue par morceaux sur $I$ et si l'on se trouve dans l'une des deux situations suivantes\footnote{En pratique, on montre seulement la condition (2). Si I est un segment, alors l'intégrale de $f$ sur I est ACV.} :
\begin{enumerate}[leftmargin=6cm,label=(\arabic*)\;]
    \item $I$ est un segment.
    
    \item L'intégrale de $f$ sur $I$ est absolument convergente.
\end{enumerate}
\vspace{0.3cm}

On note \(L^1\bigl(I,\KK\bigr)\) l'ensemble des applications intégrables sur $I$.

\vspace{1.5cm}

Soit \(\,f:I\to \KK\,\) \underline{intégrable} sur $I$.\vspace{-0.2cm}\\
Si \(\,I=[a,b]\,\) avec \(\,-\infty<a\leq b<+\infty\,\) alors on pose : \(\!\displaystyle \int_{I}f=\int_{a}^{b}\!f\;\) où \(\,\displaystyle\int_{a}^{b}\!f\,\) est l'intégrale de $f$ sur le segment \([a,b]\).\vspace{0.1cm}\\
Si $I$ est un intervalle semi-ouvert ou ouvert d'extrémités $a$ et $b$ avec \(\,-\infty\leq a <b\leq+\infty\,\) alors on pose \(\!\displaystyle \int_{I}f=\int_{a}^{b}\!f\;\) où \(\,\displaystyle\int_{a}^{b}\!f\,\) est l'intégrale généralisée de $f$ sur $I$.\vspace{0.1cm}\\
Dans tous les cas, le scalaire \(\,\displaystyle\int_I f\,\) est appelé \textbf{intégrale de $f$ sur I.}\index{intégrale d'une fonction sur un intervalle}

\newpage

\textbf{Relations de Chasles}\index{relations de Chasles (intégrales)} : \vspace{0.2cm}\\
Soit \(f\in L^1\bigl(I,\KK\bigr)\,\) et \(\,(a,b)\in\overline{\RR}^2\:\) tel que \(\,]\min(a,b),\,\max(a,b)\,[\ \subset I.\)
\begin{itemize}[leftmargin=1.5cm]
    \item[•] Si \(\,-\infty\leq a<b \leq+\infty\,\) alors \(\,f\in L^1\bigl(\,]a,b[\,,\KK\bigr)\,\) et on pose \(\,\displaystyle \int_{a}^{b}\!f=\int_{]a,b[}f\)
    
    \item[•] Si \(\,-\infty\leq b<a \leq+\infty\,\) alors \(\,f\in L^1\bigl(\,]b,a[\,,\KK\bigr)\,\) et on pose \(\,\displaystyle \int_{a}^{b}\!f=-\!\int_{]b,a[}f\)\vspace{-0.2cm}
    
    \item[•] Si \(\,-\infty\leq a \leq+\infty\,\) alors on pose \(\,\displaystyle\int_{a}^{a}\!f=0\)
\end{itemize}

\vspace{1.5cm}

\textbf{Intégrale dépendant d'un paramètre :}\index{intégrale dépendant d'un paramètre}\vspace{0.2cm}\\
On suppose $I$ d'intérieur non vide. Soient A une partie de $\,\RR\,$ et \(\,f:A\times I\to \KK\,\) une application.\vspace{0.1cm}\\
Pour \(x\in A\) on note \(f(x,\sbullet[0.6]\, )\) l'application de $I$ dans $\,\KK\,$ définie par : \(f(x,\sbullet[0.6]\,)(t)=f(x,t)\).\vspace{0.1cm} \\
Pour \(t\in I\) on note \(f(\,\sbullet[0.6]\,,t)\) l'application de A dans \(\,\KK\,\) définie par : \(\,f(\,\sbullet[0.6]\,,t)(x)=f(x,t).\)\vspace{0.3cm}\\
Si pour tout \(x\in A\) l'application \(f(x,\,\sbullet[0.6])\) est intégrable sur $I$ alors on dispose de l'application\vspace{0.1cm}\\
\(F:A\to \KK\,\) définie par : \(\,\displaystyle F(x)=\int_I f(x,t)\,dt=\int_If(x,\,\sbullet[0.6])\)

\vspace{1.5cm}

\subsection{Étude locale}

\vspace{0.5cm}

\begin{center}
    Soient A une partie de \(\,\RR\,\), \(a\in \overline{\text{A}}\:\) et \(\,f,\,g\,:\RR\to \KK.\)
\end{center}

\vspace{1cm}

On dit que $f$ \guillemetleft \emph{est d'un certain type}\guillemetright\, au voisinage de $a$ dans A \ssi il existe un voisinage V de $a$ dans \(\,\overline{\RR}\,\) tel que $f$ \guillemetleft \emph{soit d'un certain type}\guillemetright\, dans V\(\,\cap\,\)A.

\vspace{1.3cm}

\noindent On suppose que $g$ \underline{ne s'annule pas} au voisinage de $a$ dans A.
\begin{itemize}
    \item[•] On dit que $f$ est \textbf{dominée}\index{fonctions à valeurs scalaires!dominée} par $g$ au voisinage de $a$ dans A, et on note \(\,f\underset{a}{=}\mathrm{O}(g)\), \ssi \(\displaystyle\,\frac{f}{g}\,\) est bornée au voisinage de $a$ dans A.\vspace{0.2cm}
    
    \item[•] On dit que $f$ est \textbf{négligeable}\index{fonctions à valeurs scalaires!négligeable} devant $g$ au voisinage de $a$ dans A, et on note \(\,f\underset{a}{=}\mathrm{o}(g)\), \ssi \(\ \displaystyle \lim_{a,\,A}\,\frac{f}{g}=0\)\vspace{0.2cm}
    
    \item[•] On dit que $f$ est \textbf{équivalente}\index{fonctions à valeurs scalaires!équivalentes} à $g$ au voisinage de $a $dans A, et on note \(\,f \underset{a}{\sim}g\),\, \ssi \(\ \displaystyle \lim_{a,\,A}\, \frac{f}{g}=1\)
\end{itemize}
\vspace{0.3cm}
\begin{small}
    \noindent La notation \(\,f\underset{a}{=}\mathrm{O}(g)\,\) doit se lire \guillemetleft $\,f$ est dominée par $g\,$\guillemetright\, ou \guillemetleft $\,f$ est \underline{un} grand O de $g\,$\guillemetright\,, le \guillemetleft voisinage de $a$ dans A\guillemetright\, étant sous-entendu. Il ne s'agit pas d'égalités au sens habituel mais \guillemetleft d'égalités\guillemetright\;qui se lisent de la gauche vers la droite. On peut avoir \(\,f=\mathrm{O}(h),\ g=\mathrm{O}(h)\,\) et \(\,f\neq g\).
\end{small}

\vspace{1.2cm}

Étant donné \(n\in\NN\),\, on dit que $f$ admet un \textbf{développement limité}\index{développement limité} à l'ordre $n$ au voisinage de $a$ dans A \ssi il existe $n+1$ éléments \(\, a_0,\cdots,a_n\,\) de \(\,\KK\,\) tels qu'au voisinage de $a$ dans A on ait :\vspace{-0.3cm}

\[f(x)\,\underset{a}{=}\,a_0+a_1(x-a)+a_2(x-a)^2+\cdots+a_n(x-a)^n+\mathrm{o}\Bigl((x-a)^n\Bigr)\]

Si c'est le cas, le polynôme \(\,a_0+a_1(X-a)+a_2(X-a)^2+\cdots+a_n(X-a)^n\,\) est appelé \textbf{partie régulière}\index{partie régulière d'un DL} du $dl_n(a)$\footnote{Contraction de \textit{développement limité à l'ordre $n$ au voisinage de $a$ dans A}}.

\vspace{1.3cm}

Étant donné \(n\in\NN\),\, on dit que $f$ admet un \textbf{développement asymptotique}\index{développement asymptotique} à l'ordre $n$ au voisinage de $a$ dans A \ssi il existe $n+1$ éléments \(\, c_0,\cdots,c_n\,\) de \(\,\KK\,\) et $n+1$ fonctions \(\,h_0,\cdots,h_n\,\) de \(\,\RR\,\) dans \(\,\KK\,\) définies au voisinage de $a$ tels qu'au voisinage de $a$ dans A on ait :\vspace{-0.3cm}

\[f\,\underset{a}{=}\,c_0\,h_0+c_1\,h_1+c_2\,h_2+\cdots+c_n\,h_n+\mathrm{o}\bigl(h_n\bigr)\]

\vspace{1.5cm}

\subsection{Équation différentielle linéaire}

\vspace{0.7cm}

\begin{center}
    I désigne un \underline{intervalle} de \(\,\RR\).
\end{center}

\vspace{0.3cm}

\subsubsection[EDL d'ordre 1]{Équation différentielle linéaire d'ordre 1}

\vspace{0.7cm}

Soient $\,a,\,b\,:I\to \KK\,$ des applications \underline{continues} sur I et à valeurs dans $\,\KK$.\vspace{0.2cm}\\
On considère l'\textbf{équation différentielle linéaire d'ordre un}\index{équation différentielle linéaire d'ordre un} : \(\, \left(\mathsf{L}\right)\,:\; y'+a\,y=b.\)\vspace{0.2cm}\\
À l'équation $\left(\mathsf{L}\right)$ on associe l'\textbf{équation différentielle homogène}\index{équation différentielle linéaire!équation différentielle homogène} : \(\, \left(\mathsf{H}\right)\,:\; y'+a\,y=0.\)

\vspace{1.2cm}

Une application \(\,f:I\to \KK\,\) est dite \textbf{solution}\index{équation différentielle linéaire!solution} de l'équation différentielle $\left(\mathsf{L} \right)$ sur l'intervalle I si\vspace{0.1cm}\\
et seulement si $f$ est dérivable sur I et si \(\, \forall t \in I,\ \;f'(t)+a(t)f(t)=b(t)\).\vspace{0.3cm}\\
On note \(\,S_{_{I\to\KK}}(\mathsf{L})\,\) l'ensemble des applications \(\,f:I\to \KK\,\) qui sont solutions de $\left(\mathsf{L}\right)$ sur I.


\vspace{1cm}

\subsubsection[EDL d'ordre 2]{Équation différentielle linéaire d'ordre 2}

\vspace{1cm}

Soient \(\,\alpha,\, \beta\,\) des éléments de \(\,\KK\,\) et \(\,c:I\to\KK\,\) une application \underline{continue} sur I à valeurs dans \(\,\KK.\)\vspace{0.2cm}\\
On considère l'\textbf{équation différentielle linéaire du second ordre à coefficients constants}\index{équation différentielle linéaire!équation différentielle linéaire du second ordre à coefficients constants} :\vspace{-0.3cm}

\[(\mathsf{L})\,:\;y''+\alpha\,y'+\beta\,y=c\]

\vspace{0.3cm}

À l'équation $\left(\mathsf{L}\right)$ on associe l'\textbf{équation différentielle homogène}\index{équation différentielle linéaire!équation différentielle homogène} : \(\,(\mathsf{H})\,:\; y''+\alpha\,y'+\beta\,y=0\)

\vspace{1.2cm}

Une application \(\,f:I\to \KK\,\) est dite \textbf{solution}\index{équation différentielle linéaire!solution} de l'équation différentielle $\left(\mathsf{L} \right)$ sur l'intervalle I si\vspace{0.1cm}\\
et seulement si $f$ est deux fois dérivable sur I et si \(\, \forall t \in I,\ \;f\,''(t)+\alpha f\,'(t)+\beta f(t)=c(t)\).\vspace{0.3cm}\\
On note \(\,S_{_{I\to\KK}}(\mathsf{L})\,\) l'ensemble des applications \(\,f:I\to \KK\,\) qui sont solutions de $\left(\mathsf{L}\right)$ sur I.

\vspace{2cm}
