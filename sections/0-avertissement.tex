\vspace{2cm}

\noindent\begin{Large} \textbf{\underline{Avertissement}}\end{Large} (surtout destiné aux 3/2) : Ce document n'est \underline{\textbf{pas}} un support d'apprentissage.

\vspace{3cm}

Lorsque tu parcourras ces pages, dès que tu tomberas sur un terme qui n'a pas encore été vu en cours : \textbf{ne t'attarde pas sur la définition}. 

\vspace{0.5cm}

À aucun moment durant l'année il ne te sera demandé de maîtriser tous ces concepts, mais lors des concours...

\vspace{0.5cm}

Le nombre de définitions peut paraître impressionnant \begin{small}(car oui, il y en a énormément)\end{small} mais des générations d'étudiants ont réussi avant toi, beaucoup qui sont bien moins fort que toi, alors n'aies pas d'inquiétude !

\vspace{3cm}

\underline{\emph{Utilisation recommandée}} : Télécharges ce pdf sur ton téléphone (ou ordinateur, tablette, etc...) afin qu'il soit rapidement accessible.\\
Dès que tu te heurtes à un oubli, partiel ou total, d'un terme \begin{small}(lors de la lecture d'un énoncé par exemple)\end{small} : cherche le via l'outil de recherche (\emph{la loupe} sur téléphone, \emph{ctrl+f} sur ordinateur) afin d'immédiatement lever le mystère.\vspace{0.2cm}\\
Tu peux également vérifier tes connaissances lors de tes révisions en te \guillemotleft testant\guillemotright.

\vspace{6cm}

\begin{center}\noindent Si tu repères une faute, si tu as une question ou une recommandation, n'hésite pas à me contacter par mail\footnote{C'est une vraie adresse sur laquelle je suis joignable, elle est simplement sécurisée via \emph{Firefox Relay}. Tu peux la sélectionner pour copier/coller.} :\hspace{0.3cm} \textbf{6clq2lpj8@mozmail.com} \end{center}

\vspace{1cm}